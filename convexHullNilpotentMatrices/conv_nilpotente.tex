\documentclass[10pt]{article}
\usepackage[francais]{babel}
\usepackage[utf8]{inputenc}
\usepackage{amsfonts, amsmath, amssymb}
\usepackage[dvips]{graphicx}
\usepackage[a4paper,vmargin=20mm,hmargin=20mm]{geometry}
\pagestyle{empty}

\newcommand{\R}{\mathbb{R}}
\newcommand{\norm}[1]{\ensuremath{\Arrowvert #1 \Arrowvert}}
\newcommand{\trans}[1]{\ensuremath{{#1}^{\!\top}}}
\newcommand{\eps}{\varepsilon}
\newcommand{\N}{\mathbb{N}}
\newcommand{\ra}{\rightarrow}
\newcommand{\conv}{\mathrm{conv} \;}
\newcommand{\tr}{\mathrm{tr} \;}
\newcommand{\matr}[2]{\R^{#1 \times #2}}
\newcommand{\nilp}{\mathcal{N}}
\newcommand{\trnul}{\mathrm{Vect}\{{\bf I}\}_n^{\bot}}
\newcommand{\tran}{^{\top}}
\newcommand{\ei}[1]{{\bf e}_{#1}}
\newcommand{\Eij}[1]{{\bf E}_{#1}}
\newcommand{\psf}[2]{\ll #1, #2 \gg}
\begin{document}
\begin{center}
{\bf \Large $ \conv \nilp = \trnul $}
\end{center}
le but de cette présentation est de démontrer l'égalité suivante
$$
\conv \{ M \in \matr{n}{n} \;:\; \exists k \in \N \, , \, \\ M^k = 0\} = \{A \in \matr{n}{n} : \tr A = 0  \}
$$
Avant de commencer la démonstration, on definit quelques notations

\begin{itemize} 
        \item l'ensemble $\nilp$ des matrices nilpotentes
$$
           \nilp =  \{ M \in \matr{n}{n} \;:\; \exists k \in \N \, , \, \\ M^k = 0\}
$$
   \item et l'ensemble $\trnul$ des matrices à trace nulle - orthogonal de la droite engendrée par la matrice identité pour le produit scalaire 
       $\psf{x}{y} = \tr (x\tran y)$
$$
\trnul = \{A \in \matr{n}{n} : \tr A = 0  \}
$$
\end{itemize}

\section{Inclusion facile}

Pour démontrer, l'égalité des ensembles on va demontrer les deux inclusions. On commence par la plus simple
$$
\conv \nilp \subset \trnul
$$
si $x \in \conv \nilp $ alors $\exists \alpha \in [0,1]$ et $u,v \in \nilp$ tels que
$$
x = \alpha u + (1-\alpha) v
$$
Puisque $u \in \nilp$ alors $\exists k$ tel que $u^k = 0$. On en déduit que les valeurs propres de u sont nulles et donc leur somme aussi.
Donc
$$
\tr u = 0  \mbox { et pour les mêmes raisons } \tr v = 0
$$
donc
$$
\tr x = 0
$$
\section{Inclusion plus difficile}

\subsection{$\conv \nilp$ est un espace vectoriel}
\begin{itemize}
    \item $\nilp$ est stable par multiplication par un scalaire, en effet soit $\lambda \in \R$
$$
A^k = 0 \Rightarrow \lambda^k A^k = 0 \Rightarrow \lambda A \in \nilp
$$
  il en va de même pour son enveloppe convexe
$$
\lambda \dot ( \alpha u + (1-\alpha) v ) = \alpha \underbrace{\lambda u}_{\in \nilp}  + (1- \alpha) \underbrace{\lambda v}_{\in \nilp}
$$
   \item $\conv \nilp $ est stable par addition :  soient $x,y \in \conv \nilp$ , comme $ \conv \nilp$ est convexe et d'interieur non vide
$$
        x + y = 2 \left(\frac{1}{2} x + \frac{1}{2} y \right)  \in \conv \nilp
$$
\end{itemize}

\subsection{base de $\trnul$}

On sait que $\conv \nilp \subset \trnul$, donc $\dim \conv \nilp \leqslant \dim \trnul$, or par defition du noyau d'une forme linéaire 
on a
$$
\dim \trnul = \dim \matr{n}{n} - 1  = n^2  - 1
$$
Supposons que l'on trouve une famille libre d'éléments de $\trnul$ de cardinal $n^2 - 1 $ -  donc une base de $\trnul$ - dont chaque élément
soit contenu dans $\conv \nilp$, alors nécéssairement on aura l'égalité des sous-espaces

Soit $(\ei{i})_{i=1\cdots n}$ les vecteurs de la base canonique de $\R^n$, on rappelle que les
$$
\Eij{ij} = \ei{i} \ei{j} \tran \mbox{ pour } i,j \in \{1 \cdots n\}^2
$$
forment une base de $\matr{n}{n}$  et la règle de calcul suivante
$$
\Eij{ij}\Eij{kl} = \delta_{jk}\Eij{il}
$$
Alors on s'interesse à la famille
$$
\mathcal{F} = \left(\Eij{ij}\right)_{i \neq j} \cup \left(\Eij{11}-\Eij{ii}\right)_{i=2\cdots n}
$$
de cardinal $n(n-1) + n-1 = n^2 -1$

\subsubsection{$\mathcal{F}$ est libre}

Soit $(\lambda_{ij}, \mu_k)_{ i \neq j, k=2\cdots n}$ des réels tels que 

$$
 \sum_{\{i,j : i \neq j\}} \lambda_{ij} \Eij{ij} + \sum_{k=2\cdots n} (\Eij{11} - \Eij{kk}) \mu_k = 0
$$
alors on va montrer que nécessairement les $\lambda$ et les $\mu$ sont nuls.\ 

soit $(r,s)$ tel que $r \neq s$ alors en multipliant à droite par $\Eij{rs}$ on obtient que 
$$
 \sum_{\{i,j : i \neq j\}} \lambda_ij \delta_{jr} \Eij{is} + \sum_{k=2\cdots n} (\delta_{1r}\Eij{1s} - \delta_{kr}\Eij{ks}) \mu_k = 0
$$
qui se réduit à 
$$
 \sum_{i} \lambda_{ir} \Eij{is} = 0
$$
car $r = 1$ et $k = r$ avec $k > 2$ est impossible
Comme les $\Eij{ij}$ forment une famille libre, toute sous famille est libre et nécessairement $\lambda_{ir} = 0$. Pour chaque $i$ donné,
en choissisant $r \neq i $, on en déduit que les $\lambda$ sont tous nuls. Donc il reste dans la somme
$$
 \sum_{k=2\cdots n} (\Eij{11} - \Eij{kk}) \mu_k = 0
$$
Si on multiplie par $\Eij{ll}$ pour $l \geqslant 2$, on a
$$
 \sum_{k=2\cdots n} (\delta_{1l}\Eij{1l} - \delta_{kl}\Eij{kl}) \mu_k = 0
$$
qui se réduit à
$$
 (\Eij{11} - \Eij{ll}) \mu_l = 0
$$
comme $l \geqslant 2$, on a clairement $\mu_l = 0$ pour $l \geqslant 2$.

\subsubsection{$\mathcal{F} \subset  \conv \nilp$}

 \begin{itemize}
     \item si $M = \Eij{ij}$ avec $i \neq $, il est trivial que que $M \in nilp$, i.e.
$$
\Eij{ij}\Eij{ij} = \delta_{ij} E_{ij} = 0 
$$
car $i \neq j$. 
  \item si $M = \Eij{11} - \Eij{kk} $ avec $ k \geqslant 2$, alors posons

$$
\xi_k = \Eij{11} - \frac{1}{2} \Eij{1k} + 2 \Eij{k1} -\Eij{kk}
$$
on vérifie que
$$
\xi_k^2 = \Eij{11} - \frac{1}{2} \Eij{1k} - \Eij{11} + \frac{1}{2} \Eij{1k} + 2 \Eij{k1} - \Eij{kk} - 2\Eij{k1} + \Eij{kk} = 0
$$
donc $\xi_k \in \nilp$
de même
$$
\zeta_k = \Eij{11} + \frac{1}{2} \Eij{1k} - 2\Eij{k1} -\Eij{kk} 
$$
verifie
$$
\zeta_k^2 = \Eij{11} + \frac{1}{2} \Eij{1k} -\Eij{11} -\frac{1}{2} \Eij{1k} - 2\Eij{k1} - \Eij{kk}  + 2 \Eij{k1} + \Eij{kk} = 0
$$
et donc $\zeta_k \in \nilp $.
alors
$$
M =  \frac{1}{2} \xi_k + \frac{1}{2} \zeta_k \in \conv \nilp
$$
 \end{itemize}

donc $\mathcal{F} \subset \conv \nilp$
\end{document}
